\documentclass[12pt]{article}

% Preamble - the area between \documentclass{...} and \begin{document}
\usepackage[T1]{fontenc}  % When you use the Palatino font, you need to use 
\usepackage{textcomp}     % packages fontenc (with option T1) and textcomp
\usepackage{palatino}
\usepackage{url}
\usepackage{fullpage}     % Use 1 inch margins

\pagenumbering{gobble}    % Supressing page numbers

% Preamble - the area between \documentclass{...} and \begin{document

% Document Body - the area between \begin{document} and \end{document}
\begin{document}

\section*{Some Interesting \TeX\ and \LaTeX\ Extras} 

This excerpt was modified from `A Simplified Introduction to \LaTeX\, which you can download from \url{https://ctan.org/pkg/simplified-latex}:

\begin{figure}[ht]
{\small
\[ \begin{array}{|l|l} \cline{1-1}
     \begin{array}{l}
        \verb|% This is myfile.tex| \\
        \verb|% |\mbox{\textit{notes to yourself can go here}}
      \end{array}
                           & \left\} \begin{array}{l} \\
                                  \mbox{Anything following \% is ignored} \\
                                  \mbox{ (used for {comments}).}
                                \end{array}\right.
 \\
     \begin{array}{l}
         \verb|\documentclass[|options\verb|]{|style\verb|}| \\
           \mbox{ \it optional specifications} \\
           \mbox{ --- e.g., declaring use of packages } \\
     \end{array}
                                       & \left\} \begin{array}{l}
                                            \mathit{Preamble}\index{preamble} \\
                                      \mbox{(blank lines do not matter)} \\
                                           \end{array}\right.
 \\ \\
    \begin{array}{l}
       \verb|\begin{document}| \\ {\vdots} \\
       \verb|\end{document}| \\
    \end{array}
                                  & \left\} \begin{array}{l}
                                        \mathit{Body}\index{body} \\
                                        \mbox{This is the document \it{envi
ronment}.}
\vspace{.06in} \\
                                      \end{array} \right.
\\                                &  \begin{array}{l}
                                        \mbox{All that follows is ignored}\\
                                        \mbox{(could be used for comments).}
                                      \end{array}

\\ \cline{1-1}
   \end{array}
\]
} % end small
\vspace{-.2in}
\caption{The Structure of a \LaTeX\ Document. \label{fig:document}}
\end{figure}


\subsection*{More information}

The following examples were taken or modified from `TeX for the Impatient', which you can download from \url{https://ctan.org/pkg/impatient}.


\subsection*{Examples}

Showing \verb|{\char}| command: {\char65}

Some math: $ 2+2=5 $

\par 

{\fontfamily{phv}\selectfont
Helvetica looks like this}.
{\fontencoding{OT1}\fontfamily{bch}\selectfont
Charter looks like this}.

\par

\noindent Here's an example of some words in a ruled box, just as Lewis Carroll wrote them:

\bigskip

% Put 8pt of space between the text and the surrounding rules.
\hbox{\vrule\vbox{\hrule
\hbox spread 8pt{\hfil\vbox spread 8pt{\vfil
\hbox{Who would not give all else for twop}%
\hbox{ennyworth only of Beautiful Soup?}%
\vfil}\hfil}
\hrule}\vrule}%

\bigskip

{\hfil\hbox to 3in{\leaders\hbox{ * }\hfil}\hfil}

\bigskip

\baselineskip = 11pt \lineskiplimit = 1pt
\lineskip = 2pt plus .5pt
Sometimes you'll need to typeset a paragraph that has
tall material, such as a mathematical formula, embedded
within it. An example of such a formula is $n \choose k$.
Note the extra space above and below this line as
compared with the other lines.
(If the formula didn't project below the line,
we'd only get extra space above the line.)

\bigskip

\centerline{$\Downarrow$}\kern 3pt      % a vertical kern
\centerline{$\Longrightarrow$\kern 6pt  % a horizontal kern
{\bf Heed my warning!}\kern 6pt         % another horizontal kern
$\Longleftarrow$}
\kern 3pt                               % another vertical kern
\centerline{$\Uparrow$}

% A small font and close interline spacing make this work
\smallskip\font\sixrm=cmr6 \sixrm \baselineskip=7pt
\dimen0=\fontdimen3\font \dimen2=\fontdimen4\font
\fontdimen3\font=1.8pt \fontdimen4\font=.9pt
\noindent \hfuzz=.1pt
\parshape 30 0pt 120pt 1pt 118pt 2pt 116pt 4pt 112pt 6pt
108pt 9pt 102pt 12pt 96pt 15pt 90pt 19pt 84pt 23pt 77pt
27pt 68pt 30.5pt 60pt 35pt 52pt 39pt 45pt 43pt 36pt 48pt
27pt 51.5pt 21pt 53pt 16.75pt 53pt 16.75pt 53pt 16.75pt 53pt
16.75pt 53pt 16.75pt 53pt 16.75pt 53pt 16.75pt 53pt 16.75pt
53pt 14.6pt 48pt 24pt 45pt 30.67pt 36.5pt 51pt 23pt 76.3pt
The wines of France and California may be the best known,
but they are not the only fine wines. Spanish wines are
often underestimated, and quite old ones may be available at
reasonable prices. For Spanish wines the vintage is not so
critical, but the climate of the Bordeaux region varies
greatly from year to year. Some vintages are not as good as
others, so these years ought to be s\kern -.1pt p\kern -.1pt
e\kern -.1pt c\hfil ially n\kern .1pt o\kern .1pt
t\kern .1pt e\kern .1pt d\hfil: 1962, 1964, 1966. 1958,
1959, 1960, 1961, 1964, 1966 are also good California
vintages. Good luck finding them!
\fontdimen3\font=\dimen0 \fontdimen4\font=\dimen2

\bigskip

\vbox to 1pc{\hrule width 6pc           % Top of box.
\hbox{1} \vskip 1pc\hbox to 2pc{\hfil 2}
% The \vss absorbs the extra distance produced by \vskip.
\vss \hbox to 3pc{\hfil 3}
\hrule width 6pc}                       % Bottom of box.

\bigskip

\def\a{\hbox to 1pc{\hfil 2}\vfil}
\vbox to 4pc{\hbox{1} \vfil \a
\vfilneg \hbox to 2pc{\hfil 3}}

\bigskip

\hbox{ugly suburban sprawl}
\hbox to 2in{ugly \hfil suburban \hfil sprawl}
\hbox spread 1in {ugly \hfil suburban \hfil sprawl}
% Without \hfil in the two preceding lines,
% you'd get `underfull hbox'es.

\bigskip

\hbox{\hsize = 10pc \raggedright\parindent = 1em
\vtop{In this example, we see how to use vboxes to
produce the effect of double columns. Each vbox
contains two paragraphs, typeset according to \TeX's
usual rules except that it's ragged right.\par
This isn't really the best way to get true double
columns because the columns}
\hskip 2pc
\vtop{\noindent
aren't balanced and we haven't done anything to choose
the column break automatically or even to fix up the
last line of the first column.\par
However, the technique of putting running text into a
vbox is very useful for placing that text where you
want it on the page.}}

\bigskip

\emph{Manipulating boxes}

\hbox{\hsize = 1in \raggedright\parindent = 0pt
\vtop to .75in{\hrule This box is .75in deep. \vfil\hrule}
\qquad
\vtop{\hrule This box is at its natural depth. \vfil\hrule}
\qquad
\vtop spread .2in{\hrule This box is .2in deeper than
its natural depth.\vfil\hrule}}

\bigskip

\emph{Retrieving the contents of a box register}

\setbox0 = \hbox{mushroom}
\setbox1 = \vbox{\copy0\box0\box0}
\box1

\bigskip

\setbox0 = \hbox{good }
Have a \copy0 \box0 \box0 day!

\bigskip

% \ht 〈register〉 [ 〈dimen〉 parameter ]
% \dp 〈register〉 [ 〈dimen〉 parameter ]
% \wd 〈register〉 [ 〈dimen〉 parameter ]
% These parameters refer to the height, depth, and width respectively of
% box register 〈register〉. You can use them to find out the dimensions of
% a box. You can also change the dimensions of a box, but it’s a tricky
% business; if you want to be adventurous you can learn all about it from
% pages 388–389 of The TeXbook.
\setbox0 = \vtop{\hbox{a}\hbox{beige}\hbox{bunny}}%
The box has width \the\wd0, height \the\ht0,
and depth \the\dp0.

\hrule\smallskip
\hrule width 2in \smallskip
\hrule width 3in height 2pt \smallskip
\hrule width 3in depth 2pt

% Here you can see how the baseline relates to the
% height and depth of an \hrule.
\leftline{
\vbox{\hrule width .6in height 5pt depth 0pt}
\vbox{\hrule width .6in height 0pt depth 8pt}
\vbox{\hrule width .6in height 5pt depth 8pt}
\vbox{\hbox{ baseline}\kern 3pt \hrule width .6in}
}

\hbox to 3in{\vrule \rightarrowfill \ 3 in
\leftarrowfill\vrule}

{\hsize=1in \parindent=0pt
\valign{#\strut&#\strut&#\strut&#\strut\cr
\noalign{\vrule width 2pt\quad}
bernaise&curry&hoisin&hollandaise\cr
\noalign{\vrule width 2pt\quad}
ketchup&marinara&mayonnaise&mustard\cr
\noalign{\vrule width 2pt\quad}
rarebit&tartar\cr
\noalign{\vrule width 2pt\quad}}}

$$\hbox to 1in{\downbracefill} \quad
\hbox to 1in{\upbracefill}$$

$$\cdots \Big\arrowvert \cdots \Big\Arrowvert \cdots
\Big\lmoustache \cdots \Big\rmoustache \cdots
\Big\bracevert \cdots$$

\bigskip

\font\tentt = cmtt10
\font\bigttfont = cmtt10 scaled \magstep2
\font\eleventtfont = cmtt10 at 11pt
First we use {\tentt regular CM typewriter}.
Then we use {\eleventtfont eleven-point CM typewriter}.
Finally we use {\bigttfont big CM typewriter}.

\bigskip

Here's a line printed normally.\par
\dimen0=\fontdimen2\font
\fontdimen2\font=3\fontdimen2\font % triple word spacing
\noindent Here’s a really spaced-out line.
\fontdimen2\font=\dimen0

\bigskip

\romannumeral 24

\bigskip

\time 1 
\day 10
\month 4
\year 1054
\today

\bigskip

This book was produced with the \fmtname\ format.

\bigskip

[{\tt \meaning\eject}] [\meaning\tenrm] [\meaning Y]

the control sequence {\tt \string\bigbreak}

\font\myfive=cmr5 [\fontname\myfive]

\bigskip

\def\a{One \vbox\bgroup}
% You couldn't use { instead of \bgroup here because
% TeX would not recognize the end of the macro
\def\enda#1{{#1\egroup} two}
% This one is a little tricky, since the \egroup actually
% matches a left brace and the following right brace
% matches the \bgroup. But it works!
\a \enda{\hrule width 1in}

\bigskip

\long\def\aa#1{\par\hrule\smallskip#1\par\smallskip\hrule}
\aa{This is the first line.\par
This is the second line.}
% without \long, TeX would complain

\bigskip

\mathchardef\alphachar = "010B % As in plain TeX.
$\alphachar$

\bigskip

\def\first{abc}
\if\first true\else false\fi;
% ‘‘c’’ is left over from the expansion of \first.
% It lands in the unexecuted ‘‘true’’ part.
\if a\first\ true\else false\fi;
% Here ‘‘bc’’ is left over from the expansion of \first
\if \hbox\relax true\else false\fi
% Unexpandable control sequences test equal with ''if''

\dimen0 = 2.5in
\hbox to \dimen0{$\Leftarrow$\hfil$\Rightarrow$}

\skip2 = 2in
$\Rightarrow$\hskip \skip2 $\Leftarrow$

\bigskip

\hbox{\vrule
  \vbox{\hrule \vskip 3pt
    \hbox{\hskip 3pt
      \vbox{\hsize = .7in \raggedright
        \noindent Help! Let me out of here!}%
    \hskip 3pt}%
  \vskip 3pt \hrule}%
\vrule}

\end{document}
% ----
% Document Body - the area between \begin{document} and \end{document}
% ----
% To compile, run lualatex two times:
%   lualatex tex-latex-extras.tex 
%   lualatex tex-latex-extras.tex 
% ----
% Q: Why?
% A: Because of the following LaTeX warning, which shows after first run: 
%      Rerun to get cross-references right.
%    or:
%      No file extras.aux.
% ----
% To view:
%   zathura tex-latex-extras.pdf
% or:
%   mupdf tex-latex-extras.pdf
% ----
% To clean up (remove) all nonessential files,
% including aux, dep, dvi, postscript and pdf files:
%   latexmk -C
% ----
