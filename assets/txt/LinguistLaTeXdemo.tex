%
% Hello! Here's how this works:
%
% You edit the source code here on the left, and the preview on the
% right shows you the result within a few seconds.
%
% If you're new to LaTeX, the wikibook at
% http://en.wikibooks.org/wiki/LaTeX
% is a great place to start, and there are some examples in this
% document, too.
%
%A percent sign % makes that whole line a comment that won't affect your document
%
% Enjoy!
%
\documentclass[12pt]{article}

\usepackage[english]{babel}
\usepackage[utf8x]{inputenc}
\usepackage{amsmath}
\usepackage{graphicx}
\usepackage{hyperref}

\usepackage{tipa} %this is the package for IPA symbols. VERY IMPORTANT: always put \usepackage{tipa} BEFORE \usepackage{linguex} or else you'll get errors. 
\usepackage{linguex} %this is the package for linguist-y examples
\renewcommand{\firstrefdash}{} %I also add this part to make examples look like this: (6a) and not like this (6-a). 

\usepackage{qtree} %This is one package for making trees, although there are others


\title{Dear Linguists: Welcome to \LaTeX}
\author{by AllThingsLinguistic\thanks{LaTeX source code for this file: \url{http://db.tt/jXvJ8pRM} }}

% See the pdf version of this file here: https://www.dropbox.com/s/qj5d2b0ckcgosm8/Linguist%20LaTeX%20demo.pdf or by rendering this file in any LaTeX editor (e.g. writelatex.com)

%Everything before "begin document" is called your "preamble". It's where "usepackage" commands go. I normally just copy-paste a preamble from an existing document when making a new one.

\begin{document}
\maketitle


\section{Introduction}

This is a demo document of some of the ways LaTeX is useful for linguists. Feel free to play around editing it!\footnote{More comprehensive guide: \url{http://en.wikibooks.org/wiki/LaTeX}. Tools for linguists: \url{http://www.essex.ac.uk/linguistics/external/clmt/latex4ling/}}


\subsection{A few pointers}

The key thing to remember about LaTeX is that if you want to know how to do something, google it.

The main thing that googling will tell you is to add a specific package. This means that you should add it to your preamble (see above in the source code) along with all the other $\backslash$usepackage\{xyz\} commands. I've put some of the most common linguist packages in this document. 

\section{Making Fancy Symbols}

\subsection{IPA}

To make IPA symbols you add $\backslash$usepackage\{tipa\} to your preamble and then type your symbols using text commands. There's a whole document of them here: \url{http://www.tug.org/tugboat/tb17-2/tb51rei.pdf} or by googling TIPA LaTeX or IPA LaTeX. 

\ex. \textipa{aj hA\*rt lI\ng gwIstIks }\footnote{I'm not really sure what the vowel is in "heart". This is just a demo. Humour me.}

\ex. More symbols: \textschwa  \textcrh  \textipa{ @ B \!d \*k \;L U}

\subsection{Semantics}

You can also make semantics symbols using math mode. You need to have $\backslash$usepackage\{amsmath\} up in the preamble, and then to trigger it, you surround a command or set of commands with dollarsigns. 

\ex. $\neg \forall x [ $ LANGUAGE$(x) \to  \exists y [ $ LINGUIST$(y) \& $ STUDIES$(y)(x) ]] $

This means something like `it's not the case that for all x, if x is a language, then there exists a y such that y is a linguist and y studies x' i.e. not all languages have a linguist that studies them. You can google LaTeX math mode to learn how to make other symbols. 

\section{Making trees}

I use the package qtree for making trees. It's fairly similar to labelled bracket notation. Documentation is available here \url{http://www.essex.ac.uk/linguistics/external/clmt/latex4ling/trees/qtree/} or have a look at the source code at left. 

\ex. \Tree [.VP  \qroof{This}.NP  [.V' [.V is ] [.NP [.Det a ] [.N' [.N tree ] ] ] ] ] \label{tree}

\section{Automatic numbering and aligning}

By including the package linguex, you can get your examples to look like the kind you see in linguistics papers. (More documentation here: \url{http://texdoc.net/texmf-dist/doc/latex/linguex/linguex-doc.pdf})

The main example command is $\backslash$ex., to which you can add $\backslash$a. and so on to make sub-examples. 

\ex. This is an example.

\ex. \label{fulleg} \a. A sub-example \label{subeg}
\b. Another sub-example

\subsection{Referring to examples}

You can use the command $\backslash$label\{xyz\} to label an example or sub-example, or a section or subsection. Then if you want to refer to it later, you use $\backslash$ref\{xyz\}, where xyz is whatever name you've picked for that example. 

So, since I defined the label `tree' for the tree example in \ref{tree} above, I can refer to it using $\backslash$ref\{tree\}. 

If I add another example above \ref{tree}, then both the example number beside the tree itself and all references to it in the text will automatically update to (5). This is really useful when making multiple drafts of something.

You can refer to both main examples, like \ref{fulleg} and sub-examples, like \ref{subeg}. Give them whatever names you want. 

\subsection{Aligning 3-line glosses}

When you're writing about a language that isn't English, you may have to give both word-by-word or morpheme-by-morpheme translations as well as a free translation. 

Linguex makes it easy to automatically align the different parts of these examples, using the $\backslash$exg. command: 

\exg. Quier-o ver a mi-s amig-a-s\\
want-1SG see.INF to my-PL friend-FEM-PL\\
`I want to see my friends'

Use $\backslash$ag. $\backslash$bg. and so on for glossed sub-examples.

\section{Some general examples of other \LaTeX{} features}
\label{sec:examples}

{\it This and subsequent sections are by the creators of writelatex.com, not by AllThingsLinguistic. I've kept them in because they might be useful.}

\subsection{Sections}

Use \texttt{section}s and \texttt{subsection}s to organize your document. \LaTeX{} handles all the formatting and numbering automatically. Use \texttt{ref} and \texttt{label} for cross-references --- this is Section~\ref{sec:examples}, for example.

\subsection{Tables and Figures}

Use \texttt{tabular} for basic tables --- see Table~ref{tab:widgets}, for example. You can upload a figure (JPEG, PNG or PDF) using the files menu. To include it in your document, use the \texttt{includegraphics} command (see the comment below in the source code).

% Commands to include a figure:
%\begin{figure}
%\includegraphics[width=\textwidth]{your-figure's-file-name}
%\caption{\label{fig:your-figure}Caption goes here.}
%\end{figure}

%\begin{table}
%\centering
\begin{tabular}{l|r}
Item & Quantity \\\hline
Widgets & 42 \\
Gadgets & 13
\end{tabular}
%\caption{\label{tab:widgets}An example table.}
%\end{table}

\subsection{Mathematics}

\LaTeX{} is great at typesetting mathematics. Let $X_1, X_2, \ldots, X_n$ be a sequence of independent and identically distributed random variables with $\text{E}[X_i] = \mu$ and $\text{Var}[X_i] = \sigma^2 < \infty$, and let
$$S_n = \frac{X_1 + X_2 + \cdots + X_n}{n}
      = \frac{1}{n}\sum_{i}^{n} X_i$$
denote their mean. Then as $n$ approaches infinity, the random variables $\sqrt{n}(S_n - \mu)$ converge in distribution to a normal $\mathcal{N}(0, \sigma^2)$.

\subsection{Lists}

You can make lists with automatic numbering \dots

\begin{enumerate}
\item Like this,
\item and like this.
\end{enumerate}
\dots or bullet points \dots
\begin{itemize}
\item Like this,
\item and like this.
\end{itemize}

\end{document}